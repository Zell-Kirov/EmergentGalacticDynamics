\documentclass[12pt,a4paper]{article}
\usepackage[utf8]{inputenc}
\usepackage[T1]{fontenc}
\usepackage[english]{babel}
\usepackage{geometry}
\geometry{margin=2.5cm}
\usepackage{hyperref}
\hypersetup{
  colorlinks=true,
  linkcolor=blue,
  urlcolor=blue,
  citecolor=blue
}
\usepackage{setspace}
\onehalfspacing

\title{\textbf{Towards an Emerging Dynamics of Spiral Galaxies: \\ An Alternative to Dark Matter}}

\author{
  Gwen Mesmacre\thanks{ORCID: \href{https://orcid.org/0009-0003-1306-7036}{0009-0003-1306-7036}}\\
  Independent Researcher\\
  \texttt{gwen.mesmacre@proton.me}
}

\date{\today}

\begin{document}

\maketitle

\begin{abstract}
This document presents an alternative hypothesis to explain the flat rotation curves observed in spiral galaxies, without invoking dark matter. Instead of treating the galaxy as a sum of individual objects subject to local forces, we propose to treat it as a coherent dynamic entity, whose global structure induces an emergent force acting on the whole system. This approach relies on a conceptual change of scale, where the galaxy becomes a celestial object in its own right, endowed with specific dynamic properties. A Python simulation project is currently under development to test this hypothesis. This preliminary document aims to establish the conceptual foundation and priority of the proposed framework.
\end{abstract}

\section{Conceptual Framework and Phenomenological Explanation}

\subsection{Observed Problem}
Galactic rotation curves remain flat at large distances from the center, which contradicts the predictions of Newtonian gravity applied to visible mass.

\subsection{Proposed Hypothesis}
The galaxy should not be viewed as a collection of stars, but as a coherent structure whose global dynamics determine the motion of its components.

\subsection{Key Principle}
Once formed, the galaxy acts as an emergent dynamic entity, and not as a mere sum of local interactions. The force acting on peripheral stars is linked to the shape and cohesion of the structure, rather than to an invisible mass.

\subsection{Uniform Rotation}
The cohesion of the galactic structure induces a global dynamics that constrains rotational velocities, regardless of the local mass distribution.

\subsection{Absence of Dark Matter}
The observed phenomenon is explained by a shape force or a cohesion field, without the need to introduce undetected mass.

\subsection{Change of Scale}
By observing the galaxy as a whole, rather than as a sum of particles, a new layer of physical laws is revealed.

\section{Validation Project}

A Python project will be developed on GitHub to:
\begin{itemize}
    \item Model the emergent dynamics of a spiral galaxy.
    \item Simulate rotation curves induced by a global force acting on the structure.
    \item Compare results with observational data (LSB galaxies, M33, NGC 3198, etc.).
    \item Publish the results in a second preprint, accompanied by the source code.
\end{itemize}

\section{Objective of This Preprint}

This pre-preprint aims to:
\begin{itemize}
    \item Establish priority of the idea.
    \item Share the founding intuition of the theory.
    \item Announce the upcoming validation project.
    \item Invite reflection and collaboration.
\end{itemize}

\section*{License}
This project is licensed under the \textbf{CC BY--NC 4.0 License}.\\
\url{https://creativecommons.org/licenses/by-nc/4.0/}

You are free to copy, share, and adapt the materials, provided that:
\begin{itemize}
  \item Appropriate credit is given to the author.
  \item The materials are not used for commercial purposes without permission.
\end{itemize}

For full license details, see: \\
\url{https://creativecommons.org/licenses/by-nc/4.0/}


\vspace{1em}
\noindent \copyright~2025 Gwen Mesmacre

\end{document}